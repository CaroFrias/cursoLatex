\documentclass[a4paper,11pt]{article}
\usepackage{polyglossia}

\usepackage{xcolor}
\usepackage{listings}

\usepackage{minted}
\usepackage[colorlinks]{hyperref}

\renewcommand{\listingscaption}{Listado}
\usemintedstyle{vim}
\newmint[py]{python}{bgcolor=black}

\begin{document}

  \listoflistings
  
  \begin{lstlisting}[language=haskell, caption=Código con Listings, captionpos=b]
   -- Fibonacci!
   fib = 0 : 1 : zipWith (+) fib (tail fib)
  \end{lstlisting}
  
  \begin{listing}
   \begin{minted}[linenos,mathescape,texcl]{clojure}
   ;; Fibonacci por cortesía de \href{https://pfctelepathy.wordpress.com/}{Ekaitz}
   ;; $F_n = F_{n-1} + F_{n-2} \,/\, F_0 = 0 \wedge F_1 = 1$
   
   (defn fibo
       ([] (fibo 0 1))
       ([one two]
           (lazy-seq (cons one (fibo two (+ one two))))))
   \end{minted}
   \label{lst:fibo}
   \caption{Código con Minted}
  \end{listing}  
  
  \py|print [n for n in range(10) if n%2]|
  
  \inputminted{python}{fibonacci.py}
  
\end{document}
