\documentclass[a4paper,10pt]{book}

\usepackage{ifxetex,ifluatex}

% IDIOMA

\ifxetex
  \usepackage{polyglossia}
  \setmainlanguage{spanish}
  % Fuente
  \usepackage{fontspec}
  \setmainfont{DejaVu Serif}
  
  % Tabla en lugar de cuadro, listings como código
  \addto\captionsspanish{%
  \renewcommand{\tablename}{Tabla}%
  \renewcommand{\listtablename}{Índice de tablas}%
  }
  
\else
  \usepackage[spanish,es-tabla]{babel}
  \usepackage[utf8]{inputenc} 
  \usepackage[T1]{fontenc}
  \usepackage{lmodern}
\fi

\usepackage{amsmath, amsthm, amssymb}

% Mejora el aspecto del documento
\usepackage{microtype}
\UseMicrotypeSet[protrusion]{basicmath} % disable protrusion for tt fonts

\usepackage[usenames,dvipsnames,svgnames,table]{xcolor}

% Hipervínculos

\ifxetex
  \usepackage[setpagesize=false, % page size defined by xetex
              unicode=false, % unicode breaks when used with xetex
              xetex]{hyperref}
\else
  \usepackage[unicode=true]{hyperref}
\fi

\hypersetup{unicode=true,
            colorlinks=true,
            linkcolor=Magenta,
            citecolor=Green,
            urlcolor=Blue,
            breaklinks=true}
\urlstyle{same}  % don't use monospace font for urls

\usepackage{listings}

% Definiciones de color de http://latexcolor.com/

\definecolor{whitesmoke}{rgb}{0.96, 0.96, 0.96}
\definecolor{viridian}{rgb}{0.25, 0.51, 0.43}
\definecolor{midnightblue}{rgb}{0.1, 0.1, 0.44}
\definecolor{ballblue}{rgb}{0.13, 0.67, 0.8}
\definecolor{frenchblue}{rgb}{0.0, 0.45, 0.73}

\lstset{	
	basicstyle=\small\ttfamily,
        breaklines=true,
        columns=fixed,
        extendedchars=true,
        prebreak = \raisebox{0ex}[0ex][0ex]{\ensuremath{\hookleftarrow}},
        tabsize=2,
        backgroundcolor=\color{whitesmoke},
        keywordstyle=\bfseries\color{viridian},
        identifierstyle=\ttfamily\color{midnightblue},
        commentstyle=\itshape\color{ballblue},
        stringstyle=\color{frenchblue},        
	}

% Listings no acepta UTF8        
\lstset{literate=%
{á}{{\'a}}1
{é}{{\'e}}1
{í}{{\'i}}1
{ó}{{\'o}}1
{ú}{{\'u}}1
{Á}{{\'A}}1
{É}{{\'E}}1
{Í}{{\'I}}1
{Ó}{{\'O}}1
{Ú}{{\'U}}1
{ñ}{{\~n}}1
{ü}{{\"u}}1
{Ü}{{\"U}}1
}

\addto\captionsspanish{\renewcommand{\lstlistingname}{Código}}

\usepackage{longtable,booktabs}

\usepackage{graphicx,grffile}

\usepackage{parskip}

\title{Curso no convencional de \LaTeX}

\author{Ondiz Zarraga}

\date{\today}

\begin{document}

\maketitle

\hypersetup{linkcolor=black}


\tableofcontents
\listoftables
\listoffigures

\chapter{Introducción}
\section{¿Qué es LaTeX?}\label{que-es-latex}

LaTeX es un
\href{https://es.m.wikipedia.org/wiki/Lenguaje_de_marcado}{\emph{lenguaje
de marcado}}, es decir, es una manera de anotar un documento con su
estructura y formato. Pensad en cuando revisamos un texto, pensamos que
tal parte debe ir en negrita y que en tal otra hay que cambiar de
párrafo y escribimos unas notas en el documento para acordarnos. Un
lenguaje de marcado hace esto de manera ordenada: define diferentes
marcas para que luego el documento tome el formato adecuado al
procesarlo. Un ejemplo es el omnipresente
\href{https://es.m.wikipedia.org/wiki/HTML}{HTML}, como su propio nombre
(\emph{HyperText Markup Language}) indica. Si no sabéis de lo que hablo
dad al botón derecho y a \emph{Ver código fuente de la página}, eso que
veis es esta misma página con etiquetas que le indican al navegador cómo
la debe mostrar.

Todos los lenguajes de marcado hacen exactamente lo mismo con la
diferencia de que sus etiquetas son diferentes. Hay montones de ellos,
cada uno con su campo de aplicación, por ejemplo, HTML está enfocado a
desarrollar páginas web y LaTeX a escribir \emph{documentos serios} (que
es como nos gusta a los científicos llamar a lo que escribimos
nosotros).

¡Pero que no se me echen atrás los no científicos! Cualquiera puede
beneficiarse de un sistema que permite tener control total sobre sus
documentos. Yo lo uso hasta para escribir recetas de cocina, aunque con
truco, como contaré en futuras ediciones de este curso.

\section{Ventajas e inconvenientes}\label{ventajas-e-inconvenientes}

El motivo de usar LaTeX en la ciencia es que al haber sido inicialmente
ideado por un científico\footnote{Para los maniáticos:
\href{https://en.wikipedia.org/wiki/Donald_Knuth}{Donald Knuth}
desarrolló \href{https://en.wikipedia.org/wiki/TeX}{TeX} y más tarde
Leslie Lamport escribió un conjunto de macros para TeX a las que se
llamó \href{http://www.tex.ac.uk/FAQ-latex.html}{LaTeX}. Pero tengamos
la fiesta en paz, \emph{porfis}.}, tuvo en cuenta nuestras
necesidades. En su momento la principal era escribir ecuaciones
decentes, hoy en día a pesar de que ese sigue siendo un punto fuerte, yo
destacaría varias cosas:

\begin{itemize}
\item
  Trabajamos en \textbf{texto plano} y generamos el documento en
  \emph{pdf}/\emph{dvi}: nos podemos ir olvidando de las versiones del
  programa (¡hola, Word!), nuestra fuente siempre será accesible y
  siempre podremos leer el documento final. Tiene la ventajas añadidas
  de que nuestros archivos son ligeros y podemos tenerlos bajo control
  de versiones, algo que hay que tener en cuenta.
\item
  Su \textbf{flexibilidad}: hay tantísimos paquetes hoy en día que se
  puede hacer de todo. ¿Código multicolor automático?
  ¡\href{https://www.ctan.org/pkg/listings}{Concedido}! ¿Esquemas
  eléctricos?
  ¡\href{http://www.texample.net/tikz/examples/circuitikz/}{Ahí tienes}!
  ¿La \href{https://www.youtube.com/watch?v=wCyC-K_PnRY}{\emph{Curva del
  Dragón}}?
  ¡\href{http://tex.stackexchange.com/questions/230457/drawing-the-dragon-curve\#230504}{Va}!
\item
  Su gestión de \textbf{idiomas}: una vez que hemos configurado el tema,
  no hay problemas con los acentos ni la silabación. Podemos mezclar
  idiomas locamente sin que se desgracie nada.
\item
  Crear \textbf{índices, glosarios, bibliografía} \ldots{}
  automáticamente y, sobre todo, con facilidad. Hablaremos de ello largo
  y tendido.
\item
  Su cuidado por la \textbf{tipografía}: si hacemos las cosas bien nos
  podemos olvidar de
  \href{https://en.wikipedia.org/wiki/Kerning}{\emph{kernings}} chungos,
  \href{https://es.wikipedia.org/wiki/Viuda_y_hu\%C3\%A9rfana}{líneas
  viudas y huérfanas} o huecos gigantes en el texto justificado, LaTeX
  lo gestiona por nosotros.
\end{itemize}

Esto, por supuesto, viene a cambio de algo:

\begin{itemize}
\item
  Hay que \textbf{compilar}: no vemos al momento lo que estamos
  cambiando. Hay gente que detesta esto, son maneras de trabajar. Si
  pertenecéis a este grupo echadle un ojo a
  \href{http://www.lyx.org/}{LyX}, igual os convence.
\item
  \textbf{No es intuitivo}: nos vamos a enfrentar a un texto lleno de
  comandos que en un primer momento nos van a parecer chino. Es así, no
  nos vamos a engañar. Usando una IDE esto mejora, pero hay que
  reconocer que ser capaz de hacer un documento decente usando LaTeX
  tiene su cosa.
\item
  La \textbf{documentación}: esto es mi opinión personal, pero la
  documentación de LaTeX no está pensada para novatos. Encontraréis
  cursitos básicos con 4 cosas, cursos inabarcables con 5 millones de
  cosas y respuestas en StackOverflow que os explicarán cómo hacer lo
  mismo de 15 maneras diferentes y sin que nadie diga por qué. Uno de
  los profesores que me ``enseñó'' LaTeX a mí cuando tenías una duda te
  decía que \emph{buscases en Google} pegases lo que te saliese en el
  documento tuyo y lo fueses cambiando hasta conseguir lo que querías.
  Ya dije que esos dos pavos eran los dos peores profesores que he
  tenido en mi vida. Al menos que buscasen en
  \href{https://duckduckgo.com/api}{DuckDuckGo}.
\end{itemize}

No os asustéis, si yo he aprendido a manejarme en él (y a amarlo)
vosotros también podéis ¡y más rápido incluso!

\section{¿Es para mí?}\label{es-para-mi}

Si no tienes miedo de aprender, no te importa tener que esperar a
compilar para ver el resultado final de tu documento y, sobre todo, si
quieres escribir un documento formal, sí, lo es.

Si el hecho de pensar en compilar algo te tira para atrás y pasas de
dedicarle horas de vida a algo cuando con el Writer ya te va bien,
puedes darme un voto de confianza y seguir algunos \emph{fascículos} de
este curso, tal vez te cuente alguna cosa interesante.

\section{Bonus}\label{bonus}

¿Qué os parece si os digo que (casi)
puedo escribir en 
\href{http://daringfireball.net/projects/markdown/}{Markdown} 
y conseguir un resultado como de escribir en LaTeX?


\chapter{¿Qué necesito?}
\section{El editor y el compilador}\label{el-editor-y-el-compilador}

Antes de ponernos a hacer nada vamos a diferenciar dos cosas: el
\emph{editor} y el \emph{compilador}. Algo que para los que andáis en la
informática es megaevidente para el resto de nosotros oh mortales puede
suponer un lío bastante gordo.

Yo lo resumo así: puedes escribir tus historias de LaTeX en el Bloc de
Notas si quieres (\emph{el editor}), luego te buscas la vida para
convertirlo a algo que un ser humano pueda leer (\emph{el compilador}).

Ahora vamos a liarnos la manta. Resulta que LaTeX son unas macros
escritas para TeX\footnote{De hecho hay otras llamadas
  \href{https://en.wikipedia.org/wiki/ConTeXt}{ConTeXt} pero nos vamos a
  olvidar de ellas.} por lo que tenemos dos lenguajes de marcado que
además se pueden compilar con diferentes \textbf{compiladores}\footnote{Veréis
  a lo que yo llamo ``compilador'' por ahí también como \emph{LaTeX
  engine}}. Aquí tenéis un resumen rápido:

\begin{itemize}
\item
  \texttt{tex} y \texttt{latex}: compilan respectivamente TeX y LaTeX a
  dvi. Para los siguientes el que solo contenga \texttt{tex} compilará
  TeX y el que contenga \texttt{latex} compilará LaTeX
\item
  \texttt{pdftex}/\texttt{pdflatex}: compilan a pdf
\item
  \texttt{xetex}/\texttt{xelatex}: compilan a pdf pero tienen la
  diferencia que gestionan Unicode y pueden usar las fuentes del sistema
  sin necesidad de configurar nada.
\item
  \texttt{luatex}/\texttt{lualatex}: compilan a pdf. La diferencia es
  que están escritos en \href{http://www.lua.org/}{Lua}, un lenguaje de
  programación bastante interesante
\end{itemize}

Bien, ahora que sabemos de compiladores vamos a ver cómo conseguimos
nosotros tener uno que nos genere los documentos. Aquí entran las
\textbf{distribuciones} de LaTeX. Una distribución es un conjunto de
programas y paquetes que nos permiten escribir sin tener que configurar
todo a mano. Es decir, si instalamos una distribución tendremos los
compiladores de los que hablábamos antes, un gestor de paquetes y otras
cosas útiles. De los paquetes hablaremos más adelante, pero de momento
os puedo decir que las diferentes funcionalidades van en diferentes
paquetes para que podamos cargar solo las que nos interesen.

Las distribuciones más conocidas son estas:

\begin{itemize}
\item
  \href{http://www.tug.org/texlive/}{TeXLive}, distribución
  multiplataforma, la encontramos para GNU/Linux, Windows y MacOS.
\item
  \href{https://miktex.org/}{MikTeX}, una distribución específica para
  Windows
\end{itemize}

No voy a hablar de la instalación porque está más que documentada y es
sencillita (especialmente para mis hermanos linuxeros, que la tienen en
los repositorios).

Como no había suficiente locura, nos quedan los \textbf{editores}. En
sí, podemos escribir en cualquier programa pero yo personalmente no os
lo recomiendo. Al menos elegid uno que tenga sintaxis resaltada para que
no os quedéis birojos intentando descifrar qué es formato y qué
contenido. Podemos dividir los editores en dos grupos:

\begin{itemize}
\item
  \emph{Editores de propósito general}: son los que sirven para escribir
  en general. Van desde uno simple como
  \href{https://es.wikipedia.org/wiki/Gedit}{gedit} hasta bestias pardas
  como \href{http://www.vim.org/}{Vim} o mi muy amado
  \href{https://www.gnu.org/software/emacs/}{Emacs}. A nada de potente
  que sea el editor seguramente tendrá un modo o un plugin que nos
  permita compilar también.
\item
  \emph{Editores específicos}
  (\href{https://es.wikipedia.org/wiki/Entorno_de_desarrollo_integrado}{IDE}):
  son los editores desarrollados expresamente para escribir LaTeX. Hay
  bastantes, yo he usado
  \href{http://texstudio.sourceforge.net/}{TeXstudio} en Windows y
  \href{http://kile.sourceforge.net/}{Kile} en GNU/Linux, pero no por
  una razón particular.
\end{itemize}

\section{¿Qué me conviene?}\label{quuxe9-me-conviene}

Después de el rollo que os he soltado diréis ¿y ahora qué demonios uso?
¿Me conviene un IDE o no? Pues a eso no os puedo responder directamente
porque depende de vuestra manera de trabajar y vuestra experiencia, esto
es lo que yo me plantearía:

\begin{itemize}
\item
  Si ya estáis usando un editor tipo Vim o Emacs, yo miraría su modo o
  plugin correspondiente antes de nada. Así tendremos las ventajas de
  usar un editor específico y las de usar un editor general.
\item
  Si os gusta tener todo centralizado y darle solo a un botoncillo para
  que se genere el documento final, un IDE es lo vuestro.
\item
  Si os gusta tener todo bajo control, no tenéis miedo de escribir un
  Makefile y no os apetece instalar otro programa en el ordenador (¡y
  menos uno con GUI!), podéis escribir en cualquier sitio y escribir las
  órdenes de compilar a mano. Eso sí, preparaos para leer manuales a
  mansalva.
\end{itemize}

Yo tengo que reconocer que soy más de las dos primeras opciones, pero me
parece justo decir que la tercera también existe y seguro que hay gente
que la prefiere.

\section{La opción Pandoc}\label{la-opciuxf3n-pandoc}

\href{http://pandoc.org/}{Pandoc} es, aparte del programa con el mejor
nombre de la historia, un \emph{conversor de documentos}, es decir,
puede convertir documentos de un formato a otro alegremente. Podemos
usarlo para no tener que usar un IDE y que se ocupe él de compilarnos el
documento. Sobre Pandoc hablaremos en el futuro, de momento simplemente
me vale con que sepáis que existe y no vayáis por ahí diciendo que
\emph{tengo escribir en Word porque me obliga mi jefe}, tendrás que
entregarle un \emph{doc}, pero escribirlo lo escribes donde te dé la
gana, faltaría más.

Como cosa curiosa, resulta que Pandoc usa LaTeX como etapa intermedia
para pasar de Markdown a pdf con lo que podemos aprovecharnos de la
sintaxis simple de Markdown y de la potencia de LaTeX simultáneamente.
Así es como he escrito yo mi tesis, de hecho. La desventaja, claro, es
que tenemos que saber tanto LaTeX como Markdown.

\section{Recapitulación:}\label{recapitulaciuxf3n}

Resumiendo, para poder escribir cosillas en LaTeX necesitamos:

\begin{itemize}
\item
  Un \textbf{editor}, puede ser uno de uso general (como Emacs) o uno
  específico para LaTeX (como Kile). Si nuestro editor no es capaz de
  compilar directamente necesitaremos también una terminal.
\item
  Una \textbf{distribución} de LaTeX, será diferente según nuestro
  sistema operativo. La distribución incluirá diferentes compiladores.
\end{itemize}

Para la \emph{opción Pandoc} necesitamos:

\begin{itemize}
\item
  Pandoc (obviamente)
\item
  Un editor cualquiera y una terminal
\item
  Una distribución de LaTeX
\end{itemize}

\section{Referencias}\label{referencias}

\href{http://tex.stackexchange.com/questions/49/what-is-the-difference-between-tex-and-latex}{\emph{What
is the difference between TeX and LaTeX?} en StackExchange}

\href{https://en.wikibooks.org/wiki/LaTeX/Basics\#Compilation}{\emph{LaTeX/compilation}
en Wikibooks}

\href{https://en.wikipedia.org/wiki/XeTeX}{\emph{XeTeX} en la wiki}

\href{http://www.luatex.org/}{\emph{LuaTeX}}

\href{http://tex.stackexchange.com/questions/126206/why-choose-lualatex-over-xelatex\#126216}{\emph{Why
choose LuaLaTeX over XeLaTeX?} en StackExchange}

\href{http://tex.stackexchange.com/questions/13593/the-differences-between-tex-engines\#13601}{\emph{The
differences between TeX engines} en StackExchange}

\href{https://www.sharelatex.com/learn/Choosing_a_LaTeX_Compiler}{\emph{Choosing
a LaTeX Compiler}}

\href{http://www.tug.org/interest.html\#free}{\emph{Free TeX
implementations}}

\href{https://en.wikibooks.org/wiki/LaTeX/Installation}{\emph{LaTex/Installation}
en Wikibooks}

\href{https://en.wikipedia.org/wiki/Comparison_of_TeX_editors}{\emph{Comparison
of TeX editors} en la wiki}

\href{http://www.tex.ac.uk/FAQ-make.html}{\emph{Makefiles for LaTeX
documents} en \emph{UK List of TeX Frequently Asked Questions}}


\chapter{Un documento básico}
\section{Un documento y sus partes}\label{un-documento-y-sus-partes}

Un documento escrito en LaTeX tiene esta pinta\footnote{Este ejemplo
  está adaptado del
  \href{https://github.com/ekaitz-zarraga/programming-notes}{repo de
  apuntes de programación} de mi señor hermano y mío. Es una versión
  simplificada.}:

\begin{lstlisting}
\documentclass[a4paper,10pt]{article}

%%%%%%%%%%%% PREÁMBULO %%%%%%%%%%%%%%%%%%%%%

% Paquetes

\usepackage[utf8]{inputenc}
\usepackage[spanish,es-tabla]{babel}
\usepackage[T1]{fontenc}

\usepackage{listings}

% Comandos

\renewcommand{\lstlistingname}{Código}
\renewcommand{\lstlistlistingname}{Índice de fragmentos de código fuente}

% Opciones

\title{Python 2.*}
\author{Ondiz Zarraga}

%%%%%%%%%%%%%%%%%%%%%%%%%%%%%%%%%%%%%%%%%%%%%%

\begin{document}
\maketitle

\begin{abstract}
Este documento es una pequeña guía de Python 
\end{abstract}

\tableofcontents

\section{Sobre el lenguaje}

\begin{itemize}
    \item Interpretado
    \item Indentación obligatoria
    \item Distingue mayúsculas - minúsculas
    \item No hay declaración de variables (\textit{dynamic typing})
    \item Orientado a objetos  
    \item Garbage colector: quita los objetos a los que no haga referencia nada
\end{itemize}

\end{document}
\end{lstlisting}

El documento tiene dos cosas llamativas:

\begin{itemize}
\tightlist
\item
  Cosas que empiezan por contrabarra, los \textbf{comandos}
\item
  Cachos que van entre un \lstinline!\begin! y un \lstinline!\end!, los
  \textbf{entornos}
\end{itemize}

Los comandos son la manera en la que nos comunicamos con LaTeX y un
entorno es el pedazo donde se aplica un formato.

Un entorno superimportante es el entorno \lstinline!document!, ahí
dentro irá el contenido de nuestro documento. Todo lo que va entre la
definición de documento (\lstinline!\documentclass!) y el inicio del
entorno \lstinline!document! se conoce como \textbf{preámbulo} y es
donde se cargan paquetes (\lstinline!\usepackage!), se definen comandos
y se establecen opciones.

\section{Nuestro primer documento}\label{nuestro-primer-documento}

Ahora que sabemos cómo es un documento vamos a empezar a escribir uno.
El documento más básico que podemos hacer es este:

\begin{lstlisting}
\documentclass{article}

\begin{document}
Hola
\end{document}
\end{lstlisting}

Esto podemos compilarlo con el botoncillo de compilar de nuestro IDE, en
la terminar o con Pandoc.

Para compilar en la terminal haríamos (imaginemos que nuestro documento
se llama \lstinline!hola.tex!):

\begin{lstlisting}[language=bash]
pdflatex hola.tex
\end{lstlisting}

Evidentemente podéis compilar también con \lstinline!xelatex! de manera
equivalente. Si quisiéramos utilizar Pandoc tendríamos que hacer lo
siguiente:

\begin{lstlisting}[language=bash]
pandoc hola.tex -o hola.pdf
\end{lstlisting}

En los tres casos el resultado es el mismo excepto porque Pandoc no nos
deja
\href{https://ondahostil.wordpress.com/2016/11/17/lo-que-he-aprendido-archivos-auxiliares-de-latex/}{archivos
auxiliares} por ahí bailando.

¡Ya hemos escrito un documento! Debo reconocer que no es un documento
demasiado interesante, para hacer algo más chulo tenemos que aprender un
poco más de sintaxis.

\section{Un poco de sintaxis}\label{un-poco-de-sintaxis}

Para poder escribir un documento un poco más interesante tenemos que
aprendernos unos pocos comandos, hoy os voy a hablar de algunos
sencillos para que le vayáis cogiendo el truco, en las siguientes
entregas entraremos más en detalle.

Antes de nada una cosa, aun no hemos aprendido a establecer las opciones
de idioma y por lo tanto tendremos problemas en los idiomas con acentos.
Si queréis poneros a jugar ya, ponedme esto en el preámbulo, justo
debajo de \lstinline!\documentclass!:

\begin{lstlisting}
\usepackage[spanish]{babel} % sustituir spanish por el idioma
\usepackage[utf8]{inputenc}
\usepackage[T1]{fontenc}
\usepackage{lmodern}
\end{lstlisting}

En el futuro lo analizaremos de ello mejor, pero si os interesa, hablé
de ello
\href{https://ondahostil.wordpress.com/2016/10/14/lo-que-he-aprendido-idiomas-con-acentos-en-latex/}{aquí}.

\subsection{Título, capítulos y
secciones}\label{tuxedtulo-capuxedtulos-y-secciones}

Una cosa importante de LaTeX es que nos desacopla el contenido del
documento de su formato. Con esto quiero decir que nosotros le diremos
cuál es el título del documento y dónde comienzan las secciones y él les
dará el formato correspondiente según el tipo de documento y las
opciones que hayamos establecido previamente.

Pongamos por ejemplo los tipos de documento \lstinline!article! y
\lstinline!book!. Como su nombre indica, el primero se utiliza para
escribir artículos y el segundo libros. Como LaTeX es muy listo, cuando
le digamos que escriba el título, para el caso del artículo nos lo
escribirá en la parte superior de la página con el texto debajo, pero
para el caso del libro nos creará una portada. Para ambos casos la
sintaxis es exactamente la misma:

\begin{lstlisting}
\documentclass{article} % O book

% Definimos el título
\title{Título del documento}

\begin{document}
\maketitle % Creamos el título
\end{document}
\end{lstlisting}

Del mismo modo, nosotros solo le tenemos que decir el título de la
sección o el capítulo y él le dará el formato correspondiente. Otra
diferencia entre las clases \lstinline!article! y \lstinline!book! es
que \lstinline!article! no tiene capítulos, como es lógico.

Para definir capítulos y secciones utilizamos los comandos
\lstinline!\chapter! y \lstinline!\section! en el cuerpo del documento,
es decir, después de \lstinline!\begin{document}!. Por ejemplo:

\begin{lstlisting}
\chapter{Capítulo numerado}
\section{Primera sección}
\subsection{Primera subsección}

\chapter*{Capítulo sin numerar}
\section*{Sección sin numerar}
\end{lstlisting}

Como veis, podemos usar los comandos de sección y capítulo con el
asterisco para que no nos las numere. Otra cosa interesante es el grado
de anidación, tenemos secciones, subsecciones, subsubsecciones, párrafos
(\lstinline!\paragraph!) y subpárrafos (\textbackslash{}subparagraph),
cada uno con su formato definido. La clase \lstinline!book!\footnote{La
  clase \lstinline!book! no es la única que tiene partes y tal, pero de
  momento así nos vale.} también tiene por encima de las secciones,
capítulos y partes (\lstinline!\part!). Más adelante aprenderemos a
personalizar todos estos formatos porque si hay algo bueno que tiene
LaTeX es que nos deja cambiar \emph{absolutamente todo} y con relativa
facilidad (gracias a StackExchange, especialmente).

\subsection{Listas}\label{listas}

Si sois como yo os gustará especialmente este apartado: las listas.
LaTeX tiene dos tipos de listas: las numeradas y las sin numerar. Son
respectivamente los entornos \lstinline!enumerate! e
\lstinline!itemize!. Se usan exactamente igual, así que solo pongo el
ejemplo de uno de ellos:

\begin{lstlisting}
\begin{itemize}
\item Primer ítem
\item Segundo ítem
\end{itemize}
\end{lstlisting}

Lo mejor es que podemos mezclar y anidar estos dos entornos que el
simbolito y la indentación cambiarán solos sin que nos tengamos que
preocupar. Por ejemplo,

\begin{lstlisting}
\begin{enumerate}
    \item Primer ítem
    \item Segundo ítem con subítems
    \begin{itemize}
        \item Ítem sin numerar
    \end{itemize}
    \item Tercer ítem
\end{enumerate}
\end{lstlisting}

Quedará así:

\begin{center}\rule{0.5\linewidth}{\linethickness}\end{center}

\begin{enumerate}
\def\labelenumi{\arabic{enumi}.}
\tightlist
\item
  Primer ítem
\item
  Segundo ítem con subítems

  \begin{itemize}
  \tightlist
  \item
    Ítem sin numerar
  \end{itemize}
\item
  Tercer ítem
\end{enumerate}

\begin{center}\rule{0.5\linewidth}{\linethickness}\end{center}

\subsection{Imágenes}\label{imuxe1genes}

Hoy solo vamos a ver como colocar una única imagen, que lo de las
imágenes tiene un poco de lío. Lo que debemos saber es lo siguiente:

\begin{itemize}
\tightlist
\item
  Necesitamos llamar al paquete \lstinline!graphicx! en el preámbulo.
  Para eso escribiremos \lstinline!\usepackage{graphicx}! entre
  \lstinline!\documentclass! y \lstinline!\begin{document}!
\item
  Las imágenes se insertan con el comando
  \lstinline!\includegraphics[opciones]{ruta}!
\item
  Si queremos ponerles un pie de figura y una etiqueta, decidir su
  posición en la página y demás necesitamos el entorno
  \lstinline!figure!
\end{itemize}

Aquí tenemos un ejemplo de cómo insertar una imagen con en el entorno
\lstinline!figure!:

\begin{lstlisting}
\begin{figure}[h] % opción de posicionamiento
    \caption{Pie de imagen}
    \centering % imagen centrada
    % Imagen 50% de ancho del texto
    \includegraphics[width=0.5\textwidth]{ruta_a_la_imagen}
\end{figure}
\end{lstlisting}

Una cosa importante de LaTeX son los objetos \emph{flotantes}, es decir,
los que si no les obligamos, se colocan en el hueco que mejor les venga
del documento. Esto es lo que nos ocurre con las imágenes al usar el
entorno \lstinline!figure!. Nos ayuda a que no haya huecos chungos en el
documento pero a veces junta todas las imágenes en una misma página o al
final del documento. Para evitar esto tenemos las opciones de
posicionamiento, de las que hablaremos cuando profundicemos en las
imágenes.

\subsection{Tablas}\label{tablas}

Para mí las tablas son lo peor de todo LaTeX. Son la cosa menos amigable
que se puede echar uno a la cara. Con un IDE la cosa mejora, pero
imaginaros cómo será el tema que la mitad de las veces las creo online
\href{http://www.tablesgenerator.com/}{aquí} y luego pego el resultado.

Lo que debemos de saber de las tablas es lo siguiente:

\begin{itemize}
\tightlist
\item
  El contenido se especifica con el entorno \lstinline!tabular!.
  Separemos las celdas con el ampersand y cambiaremos de línea con dos
  contrabarras.
\item
  Si queremos ponerles un pie de tabla y una etiqueta, decidir su
  posición en la página y demás necesitamos el entorno \lstinline!table!
\item
  Si utilizamos el entorno \lstinline!table! la tabla se volverá
  \emph{flotante}
\end{itemize}

Aquí tenemos un ejemplo de tabla sencilla:

\begin{lstlisting}
\begin{table}
    \begin{tabular}{|ll|}
      \hline % Separador
      Columna 1 & Columna 2 \\
      1         & 2         \\
      3         & 4         \\
      \hline
    \end{tabular}
    \caption {Pie de tabla}
\end{table}
\end{lstlisting}

Al igual que con la imágenes, profundizaremos en las tablas más
adelante.

\subsection{Ecuaciones}\label{ecuaciones}

Las ecuaciones son la razón por la que mucha gente se pasa a LaTeX. Son
muy cucas y escribirlas, una vez cogido el callo, no es un infierno (de
nuevo, ¡hola, Word!). Pero no nos vamos a engañar, al principio es
\emph{muerte y destrucción}. Hay dos tipos de ecuaciones en LaTeX: las
que van dentro de la línea y las que tienen línea propia. Las primeras
van entre signos de dólar y las segundas dentro del entorno
\lstinline!equation!. Aquí tenemos un ejemplo:

\begin{lstlisting}
Imaginemos que $a=1$ en la siguiente ecuación:
\begin{equation}
ax^2 + 1 = 0
\end{equation}
\end{lstlisting}

Del mismo modo que ocurría con las secciones, si utilizamos el asterisco
la ecuación no estará numerada.

El lío con las ecuaciones es que todo se define con comandos, por
ejemplo, \lstinline!\frac{numerador}{denominador}! se usa para escribir
una fracción y \lstinline!\omega! para la letra griega omega. Los que
uséis un IDE lo tenéis más fácil porque suelen tener una barrita con los
símbolos más usados, a los demás les tocará investigar.

También tenemos otras herramientas que nos pueden ayudar a escribir
ecuaciones:

\begin{itemize}
\item
  \textbf{Editores online}: son editores con su GUI y tal que nos
  facilitan la tarea cuando todavía no conocemos los comandos de LaTeX,
  por ejemplo tenemos
  \href{http://www.numberempire.com/texequationeditor/equationeditor.php\%22}{Latex
  Equation Editor} o
  \href{https://www.latex4technics.com/}{LaTeX4technics}.
\item
  \href{http://detexify.kirelabs.org/classify.html}{\textbf{Detexify}}:
  un cacharro que nos busca el símbolo de LaTeX más parecido a algo que
  le pintemos.
\end{itemize}

Las ecuaciones se merecen una entrada propia y la tendrán.

\section{Bonus: opción Markdown}\label{bonus-opciuxf3n-markdown}

Si tenemos instalado Pandoc todo lo que hemos explicado aquí podemos
conseguirlo escribiendo en Markdown
(\href{http://rmarkdown.rstudio.com/authoring_pandoc_markdown.html}{sabor
Pandoc}) y convirtiendo a pdf. Por ejemplo, podemos meter una imagen
como:

\begin{lstlisting}
![Pie de imagen](/ruta){width=0.7 #etiqueta}
\end{lstlisting}

\section{Referencias}\label{referencias}

\href{https://en.wikibooks.org/wiki/LaTeX/Document_Structure}{\emph{LaTeX/Document
Structure} en Wikibooks}

\href{https://www.sharelatex.com/learn/Environments}{\emph{Environments}
en ShareLaTeX}

\href{http://www.personal.ceu.hu/tex/environ.htm}{\emph{Latex Standard
Environments}}

\href{http://texblog.org/2013/02/13/latex-documentclass-options-illustrated/}{\emph{LaTeX
documentclass options illustrated}}

\href{https://www.sharelatex.com/learn/Sections_and_chapters}{\emph{Sections
and chapters} en ShareLaTeX}

\href{https://en.wikibooks.org/wiki/LaTeX/Tables}{\emph{LaTeX/Tables} en
Wikibooks}

\href{https://www.sharelatex.com/learn/Inserting_Images}{\emph{Inserting
images} en ShareLaTeX}

\href{http://osl.ugr.es/CTAN/info/symbols/comprehensive/symbols-a4.pdf}{\emph{The
Comprehensive LaTeX Symbol List}}

\href{ftp://ftp.ams.org/pub/tex/doc/amsmath/short-math-guide.pdf}{\emph{Short
Math Guide for LaTeX}}


% Bibliografía

% \bibliographystyle{plain}
%\bibliography{bib}

\end{document}
