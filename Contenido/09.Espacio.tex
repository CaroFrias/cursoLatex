\section{Espacio en blanco}\label{espacio-en-blanco}

Como hemos visto en anteriores entregas, LaTeX gestiona el espacio en
blanco él solito. Esto tiene varias implicaciones:

\begin{itemize}
\item
  Le da igual que pongamos un espacio o sesenta y cuatro entre dos
  palabras, para él serán un único espacio.
\item
  Partirá las líneas donde mejor le venga a no ser que nosotros le
  obliguemos a hacerlo en un sitio determinado, dónde hayamos saltado de
  línea en el editor le da absolutamente igual. Lo mismo puede decirse
  de las páginas.
\item
  Dos trozos de texto que no estén separados por una línea en blanco
  pertenecerán al mismo párrafo a no ser que le digamos expresamente que
  no es así.
\item
  No pintará una línea en blanco a no ser que le digamos expresamente
  que lo haga por muchas líneas en blanco que tengamos en el editor.
\end{itemize}

Así visto parece que es malvado y que le gusta fastidiarnos, pero que él
tome las decisiones de formato tiene la grandísima ventaja de que
obtenemos un documento con \emph{pinta profesional} con muy poco
esfuerzo.

En cualquier caso, como nos gusta cacharrear, vamos a ver cómo menear
alegremente las cosas por el documento en contra de la lógica interna de
LaTeX. Aunque en general es mejor y más práctico dejar que LaTeX haga lo
que le dé la real gana. Avisados estáis.

\subsection{Espacio horizontal}\label{espacio-horizontal}

Hay muchas maneras de generar espacio en blanco en LaTeX, demasiadas
para mí. Os voy a hablar de las que yo uso, si queréis convertiros en
expertos en el espacio horizontal tenéis las referencias.

Primero tenemos los dos comandos que ya vimos en el capítulo de
ecuaciones:

\begin{itemize}
\item
  \lstinline!\,! crea un espacio estrecho (\lstinline!\thinspace!). Se
  usa cuando queremos separar ligeramente dos
  \href{http://defharo.com/tipografia/glifos/}{glifos} porque
  interfieren entre sí.
\item
  \lstinline!~! crea un \emph{espacio duro}, es decir, un espacio que no
  se puede partir. Es especialmente útil para las iniciales o casos como
  \emph{Ejemplo 1} donde no queremos que el uno vaya a otra
  línea\footnote{¿Recordáis el comando \lstinline!\refeq!? Lo definimos
    usando un \emph{espacio duro} por este mismo motivo:
    \lstinline!\newcommand{\refeq}[1]{Ecuación~\ref{#1}}!}.
\end{itemize}

Hay muchos más comandos de este tipo cada uno con sus reglas de uso pero
no sé cómo de útil es saberse una pila de ellos sin entender de
tipografía. Solo voy a citar la pareja \lstinline!\quad! y
\lstinline!\qquad! que equivalen respectivamente a 1 em\footnote{Un
  espacio de la anchura de una letra \emph{m} en la fuente actual} y a 2
em porque aparecen a menudo por ahí.

Por otra parte tenemos un comando que nos permite introducir un espacio
blanco del tamaño que nosotros queramos:

\begin{lstlisting}
\hspace{LONGITUD}
\end{lstlisting}

Esta longitud puede estar definida en cualquiera de las medidas que
LaTeX entiende: puntos (\emph{pt}), pulgadas (\emph{in}), centímetros
(\emph{cm}), milímetros (\emph{mm}), \emph{em}, \emph{ex}\footnote{Un
  espacio de la altura de una letra \emph{x} en la fuente actual
  {[}pica{]}: https://en.wikipedia.org/wiki/Pica\_(typography)} o
{[}picas{]}{[}pica{]} (\emph{pc}).

Una cosa a tener en cuenta es que LaTeX ignorará el
\lstinline!\hspace{}! si viene justo después de un salto de línea ya que
para él las líneas no pueden empezar con espacio en blanco. Nos ocurre
exactamente lo mismo al final de las líneas. Podemos obligarle a que
ponga siempre el espacio con la versión con asterisco
\lstinline!\hspace*{}!:

\begin{lstlisting}
% Diferencia entre \hspace y \hspace*
\hspace{1em}Línea alineada

\hspace*{1em}Línea movida 1 em hacia dentro 
\end{lstlisting}

Un comando similar a \lstinline!\hspace{}! es \lstinline!\phantom{}!,
que nos genera un espacio en blanco igual de ancho que su argumento:

\begin{lstlisting}
% Hueco del tamaño de abc
\phantom{abc}
\end{lstlisting}

Finalmente tenemos \lstinline!\hfill!, el \emph{espacio de goma}. Este
comando crea todo el espacio que puede dentro de una línea y sirve para
situar cosas espaciadas pero con un cierto orden en la línea:

\begin{lstlisting}
Texto a la izquierda \hfill Texto a la derecha
\end{lstlisting}

Todos estos comandos de espacio nos vienen bien a la hora de diseñar una
portada para nuestro documento o para crear encabezados y pies de página
personalizados.

\subsection{Espacio vertical}\label{espacio-vertical}

El espacio vertical se gestiona de manera muy similar al horizontal con
los comandos \lstinline!\vspace{}! y \lstinline!\vfill!. El primero nos
permite crear espacio vertical del tamaño deseado. Tiene una versión con
asterisco para que LaTeX no ignore el espacio vertical tras un salto de
página, es decir, al inicio de la página.

El segundo, \lstinline!\vfill!, nos crea \emph{espacio de goma} vertical
y es especialmente útil
\href{http://tex.stackexchange.com/questions/2326/vertically-center-text-on-a-page}{cuando
queremos alinear texto verticalmente}:

\begin{lstlisting}
\vfill
Texto centrado verticalmente
\vfill
\end{lstlisting}

\subsection{Párrafos y saltos de
línea}\label{puxe1rrafos-y-saltos-de-luxednea}

Hemos dicho que para LaTeX un párrafo no acaba hasta que haya una línea
en blanco o alguna indicación extra. Esta indicación es el comando
\lstinline!\par!, que apenas se usa a la hora de escribir ya que empeora
la legibilidad y se suele reservar para definir entornos.

Tenemos que tener en cuenta que solo si estamos usando el paquete
\lstinline!parskip! pintará una línea entre los párrafos, si no, por
mucho que en nuestro editor los párrafos estén separados por una o siete
líneas en blanco veremos dos párrafos juntos con el segundo de ellos
sangrado.

Para provocar saltos de línea tenemos (entre otros menos usados) dos
comandos \lstinline!\\! y \lstinline!\newline!, ambos comienzan una
nueva línea pero no un nuevo párrafo\footnote{Probad a saltar de línea
  sin usar el paquete \lstinline!parskip! ¿la línea tras el salto está
  indentada?}. El primero es preferible dejarlo para alinear, ya que
LaTeX lo redefine según el entorno, por ejemplo en las tablas, para
organizar ecuaciones con \lstinline!align! o tras un
\lstinline!centering!. El segundo solo puede usarse en el texto normal y
a ser posible sin abusar.

\subsection{Saltos de página}\label{saltos-de-puxe1gina}

Lo último que nos queda son los saltos de página. En general uno de
puede despreocupar porque LaTeX tiene la habilidad de situar las cosas
más o menos correctamente en la página. En mi opinión saber saltar de
página es necesario en tres contextos:

\begin{enumerate}
\def\labelenumi{\arabic{enumi}.}
\item
  Queremos una página en blanco al inicio del documento para poner una
  dedicatoria, un resumen, los agradecimientos o algo similar.
\item
  Nos ha empezado una sección muy abajo en la página y no nos gusta cómo
  queda.
\item
  Nos ha almacenado todas las imágenes en una página, la opción H de las
  figuras nos está haciendo cosas raras y tenemos prisa. Meter unos
  saltos de página por ahí nos puede solucionar la papeleta
  momentáneamente.
\end{enumerate}

Vamos a ver las opciones que tenemos:

\begin{itemize}
\item
  \lstinline!\newpage! provoca un salto de página dejando los párrafos
  que quedan tal cual, es decir, con espacio en la parte inferior como
  si se acabase el capítulo.
\item
  \lstinline!\pagebreak! provoca un salto de página y esparce los
  párrafos en la página para no dejar hueco blanco.
\item
  \lstinline!\clearpage! y \lstinline!\cleardoublepage! provocan un
  salto de página y hacen que se pinten todos las figuras y tablas
  definidas hasta ese punto. La diferencia entre ellos es que, como su
  nombre indica, \lstinline!\clearpage! salta una página y
  \lstinline!\cleardoublepage! salta dos caras. Si estamos en un
  documento con diferentes estilos para la página derecha y la izquierda
  \lstinline!\cleardoublepage! saltará las páginas necesarias para que
  la siguiente sea impar.
\end{itemize}

Una vez que hemos identificado los casos posibles y conocemos las
diferentes opciones vamos a combinar las dos cosas para obtener unas
\emph{reglas}:

\begin{itemize}
\item
  Para organizar secciones o figuras podemos elegir entre
  \lstinline!\newpage!, \lstinline!\pagebreak! y \lstinline!\clearpage!
  dependiendo del resultado que queramos conseguir, si es a la mitad de
  un capítulo igual nos conviene más \lstinline!\pagebreak! pero todo es
  probar.
\item
  Para las dedicatorias, resúmenes y tal de los libros la opción
  ganadora es \lstinline!\cleardoublepage! ya que nos los va a poner
  siempre a la derecha como es lo habitual.
\end{itemize}

\section{El resumen}\label{el-resumen}

Os resumo lo que hemos aprendido en este capítulo incluyendo unas
\emph{buenas prácticas}:

\begin{itemize}
\item
  \textbf{Espacio horizontal}: poner un espacio en blanco o varios entre
  dos palabras nos da exactamente el mismo resultado. Hay muchos
  comandos para crear espacio horizontal de mayor o menor tamaño,
  algunos para el texto, otros para las ecuaciones y algunos de ellos
  para ambas cosas. Si queremos crear un espacio a nuestro gusto tenemos
  \lstinline!\hspace{}! y \lstinline!\hspace*{}!, comandos que generan
  un espacio horizontal del tamaño que les pidamos, siendo el del
  asterisco imposible de ignorar. Si queremos un espacio que se ajuste
  solo tenemos \lstinline!\hfill! \emph{que es de goma}.
\item
  \textbf{Espacio vertical}: LaTeX no nos va a crear espacio blanco
  vertical a menos que se lo digamos. Para ello tenemos los comandos
  \lstinline!\vspace{}! y su versión con asterisco que fabrican espacio
  vertical del tamaño que les pidamos. Para situar elementos espaciados
  con cierto equilibrio en la página tenemos el \emph{espacio de goma}
  \lstinline!\vfill!.
\item
  \textbf{Párrafos y saltos de línea}: lo más práctico es separar los
  párrafos con una línea en blanco y solo usar los saltos de línea en
  casos especiales, para ello tenemos \lstinline!\newline! para usar en
  el texto normal y \lstinline!\\! en los entornos especiales.
\item
  \textbf{Saltos de página}: para saltar de página tenemos
  \lstinline!\newpage!, \lstinline!\pagebreak!, \lstinline!\clearpage! y
  \lstinline!\cleardoublepage!. Los tres primeros saltan a la siguiente
  página pero difieren en como tratan el texto y los objetos flotantes
  como las tablas y las figuras. El último salta dos caras y es útil
  para definir la página de dedicatoria o similares de un libro.
\end{itemize}

\section{Referencias}\label{referencias}

\href{https://www.sharelatex.com/learn/Line_breaks_and_blank_spaces}{\emph{Line
breaks and blank spaces}}

\href{http://tex.stackexchange.com/questions/74353/what-commands-are-there-for-horizontal-spacing\#74354}{\emph{What
commands are there for horizontal spacing?} en TexExchange}

\href{https://en.wikipedia.org/wiki/Whitespace_character}{\emph{Whitespace
character} en la Wikipedia}

\href{http://tex.stackexchange.com/questions/82664/when-to-use-par-and-when-or-blank-lines\#82666}{\emph{When
to use \par and when \textbackslash{}, or blank lines} en TexExchange}

\href{http://www.personal.ceu.hu/tex/breaking.htm}{\emph{LaTeX Line and
Page Breaking}}

\href{http://tex.stackexchange.com/questions/736/pagebreak-vs-newpage}{\emph{\lstinline!\pagebreak!
vs \lstinline!\newpage!} en TexExchange}
