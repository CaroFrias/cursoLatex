\section{¿Qué es LaTeX?}\label{que-es-latex}

LaTeX es un
\href{https://es.m.wikipedia.org/wiki/Lenguaje_de_marcado}{\emph{lenguaje
de marcado}}, es decir, es una manera de anotar un documento con su
estructura y formato. Pensad en cuando revisamos un texto, pensamos que
tal parte debe ir en negrita y que en tal otra hay que cambiar de
párrafo y escribimos unas notas en el documento para acordarnos. Un
lenguaje de marcado hace esto de manera ordenada: define diferentes
marcas para que luego el documento tome el formato adecuado al
procesarlo. Un ejemplo es el omnipresente
\href{https://es.m.wikipedia.org/wiki/HTML}{HTML}, como su propio nombre
(\emph{HyperText Markup Language}) indica. Si no sabéis de lo que hablo
dad al botón derecho y a \emph{Ver código fuente de la página}, eso que
veis es esta misma página con etiquetas que le indican al navegador cómo
la debe mostrar.

Todos los lenguajes de marcado hacen exactamente lo mismo con la
diferencia de que sus etiquetas son diferentes. Hay montones de ellos,
cada uno con su campo de aplicación, por ejemplo, HTML está enfocado a
desarrollar páginas web y LaTeX a escribir \emph{documentos serios} (que
es como nos gusta a los científicos llamar a lo que escribimos
nosotros).

¡Pero que no se me echen atrás los no científicos! Cualquiera puede
beneficiarse de un sistema que permite tener control total sobre sus
documentos. Yo lo uso hasta para escribir recetas de cocina, aunque con
truco, como contaré en futuras ediciones de este curso.

\section{Ventajas e inconvenientes}\label{ventajas-e-inconvenientes}

El motivo de usar LaTeX en la ciencia es que al haber sido inicialmente
ideado por un científico\footnote{Para los maniáticos:
\href{https://en.wikipedia.org/wiki/Donald_Knuth}{Donald Knuth}
desarrolló \href{https://en.wikipedia.org/wiki/TeX}{TeX} y más tarde
Leslie Lamport escribió un conjunto de macros para TeX a las que se
llamó \href{http://www.tex.ac.uk/FAQ-latex.html}{LaTeX}. Pero tengamos
la fiesta en paz, \emph{porfis}.}, tuvo en cuenta nuestras
necesidades. En su momento la principal era escribir ecuaciones
decentes, hoy en día a pesar de que ese sigue siendo un punto fuerte, yo
destacaría varias cosas:

\begin{itemize}
\item
  Trabajamos en \textbf{texto plano} y generamos el documento en
  \emph{pdf}/\emph{dvi}: nos podemos ir olvidando de las versiones del
  programa (¡hola, Word!), nuestra fuente siempre será accesible y
  siempre podremos leer el documento final. Tiene la ventajas añadidas
  de que nuestros archivos son ligeros y podemos tenerlos bajo control
  de versiones, algo que hay que tener en cuenta.
\item
  Su \textbf{flexibilidad}: hay tantísimos paquetes hoy en día que se
  puede hacer de todo. ¿Código multicolor automático?
  ¡\href{https://www.ctan.org/pkg/listings}{Concedido}! ¿Esquemas
  eléctricos?
  ¡\href{http://www.texample.net/tikz/examples/circuitikz/}{Ahí tienes}!
  ¿La \href{https://www.youtube.com/watch?v=wCyC-K_PnRY}{\emph{Curva del
  Dragón}}?
  ¡\href{http://tex.stackexchange.com/questions/230457/drawing-the-dragon-curve\#230504}{Va}!
\item
  Su gestión de \textbf{idiomas}: una vez que hemos configurado el tema,
  no hay problemas con los acentos ni la silabación. Podemos mezclar
  idiomas locamente sin que se desgracie nada.
\item
  Crear \textbf{índices, glosarios, bibliografía} \ldots{}
  automáticamente y, sobre todo, con facilidad. Hablaremos de ello largo
  y tendido.
\item
  Su cuidado por la \textbf{tipografía}: si hacemos las cosas bien nos
  podemos olvidar de
  \href{https://en.wikipedia.org/wiki/Kerning}{\emph{kernings}} chungos,
  \href{https://es.wikipedia.org/wiki/Viuda_y_hu\%C3\%A9rfana}{líneas
  viudas y huérfanas} o huecos gigantes en el texto justificado, LaTeX
  lo gestiona por nosotros.
\end{itemize}

Esto, por supuesto, viene a cambio de algo:

\begin{itemize}
\item
  Hay que \textbf{compilar}: no vemos al momento lo que estamos
  cambiando. Hay gente que detesta esto, son maneras de trabajar. Si
  pertenecéis a este grupo echadle un ojo a
  \href{http://www.lyx.org/}{LyX} o a \href{http://texmacs.org/tmweb/home/welcome.en.html}{GNU TeXmacs}\footnote{¡Gracias a Efraín Romano por la recomendación!}, 
  igual os convencen.
\item
  \textbf{No es intuitivo}: nos vamos a enfrentar a un texto lleno de
  comandos que en un primer momento nos van a parecer chino. Es así, no
  nos vamos a engañar. Usando una IDE esto mejora, pero hay que
  reconocer que ser capaz de hacer un documento decente usando LaTeX
  tiene su cosa.
\item
  La \textbf{documentación}: esto es mi opinión personal, pero la
  documentación de LaTeX no está pensada para novatos. Encontraréis
  cursitos básicos con 4 cosas, cursos inabarcables con 5 millones de
  cosas y respuestas en StackOverflow que os explicarán cómo hacer lo
  mismo de 15 maneras diferentes y sin que nadie diga por qué. Uno de
  los profesores que me ``enseñó'' LaTeX a mí cuando tenías una duda te
  decía que \emph{buscases en Google} pegases lo que te saliese en el
  documento tuyo y lo fueses cambiando hasta conseguir lo que querías.
  Ya dije que esos dos pavos eran los dos peores profesores que he
  tenido en mi vida. Al menos que buscasen en
  \href{https://duckduckgo.com/api}{DuckDuckGo}.
\end{itemize}

No os asustéis, si yo he aprendido a manejarme en él (y a amarlo)
vosotros también podéis ¡y más rápido incluso!

\section{¿Es para mí?}\label{es-para-mi}

Si no tienes miedo de aprender, no te importa tener que esperar a
compilar para ver el resultado final de tu documento y, sobre todo, si
quieres escribir un documento formal, sí, lo es.

Si el hecho de pensar en compilar algo te tira para atrás y pasas de
dedicarle horas de vida a algo cuando con el Writer ya te va bien,
puedes darme un voto de confianza y seguir algunos \emph{fascículos} de
este curso, tal vez te cuente alguna cosa interesante.

\section{Bonus}\label{bonus}

¿Qué os parece si os digo que (casi)
puedo escribir en 
\href{http://daringfireball.net/projects/markdown/}{Markdown} 
y conseguir un resultado como de escribir en LaTeX?
