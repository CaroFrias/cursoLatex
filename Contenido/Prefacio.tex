\section*{El porqué del curso}\label{sec:porque}

Publiqué la idea de escribir un cursete de LaTeX el
\href{https://ondahostil.wordpress.com/2016/11/13/proyecto-curso-no-convencional-de-latex/}{13
de noviembre del año pasado}, un mes después de terminar de escribir la
tesis y un mes antes de defenderla. Fue un tiempo muy raro ese, estaba
preparando la presentación, el ambiente en el trabajo era horrible y no
sabía qué iba a hacer con mi vida. Es fácil entender que muy motivada no
estaba.

La cuestión es que estaba haciendo la presentación con una mezcla de
Markdown, Beamer y Pandoc, siguiendo con la estela de la tesis, con la
idea de tener que aprender algo nuevo y así llevarlo mejor. Y me di
cuenta de una cosa: estaba entendiendo LaTeX. Quiero decir con esto que
podía abrir el manual de un paquete, leerlo y entenderlo.

Ya había sentido esto mismo cuando truqué un paquete para que hiciera lo
que yo necesitaba, pero ahora estaba casi convencida de ello. Además,
había logrado llegar a ese nivel de entendimiento yo sola, ya que en el
curso de LaTeX que tuve al inicio de la tesis no aprendí absolutamente
nada gracias a los magníficos profesores que lo impartían.

Para que os hagáis una idea lo que suponía para mí entender LaTeX, antes
lo que hacía era buscar en DuckDuckGo mi problema, ir a las respuestas
de StackOverflow e ir probando hasta que se solucionaba el asunto.
Tristemente no estaba sola, el resto de los doctorandos que usaban LaTeX
hacían exactamente lo mismo, precisamente porque los magníficos
profesores alentaban dicho comportamiento. El resto usaba Word porque
nadie había sido capaz de explicarles por qué LaTeX era interesante.

Me planteé por lo tanto crear el documento que me hubiera gustado tener
a mí para aprender, esa cosa intermedia entre lo más básico y leerte el
manual de 250 páginas de Beamer. Hay claramente un vacío ahí. Por eso
quería que el curso fuera relativamente ameno aunque riguroso, que
hablase de conceptos y no solo de lo que hay que poner en el .tex.. Ese
fue el motivo de que lo bautizase como no convencional.

\section*{Contenido}\label{sec:contenido}

\begin{enumerate}
\item
  \textbf{Introducción}: donde cuento qué es LaTeX y hablo de sus
  ventajas e inconvenientes.
\item
  \textbf{¿Qué necesito?}: sobre las herramientas necesaria para
  escribir un documento. Cosas sobre editores, compiladores y una
  pequeña intro a Pandoc.
\item
  \textbf{Un documento básico}: escribimos juntos nuestro primer
  documento y aprendemos sobre sintaxis.
\item
  \textbf{Insertando figuras}: sobre cómo insertar imágenes en un
  documento, objetos flotantes, formatos y posicionamiento.
\item
  \textbf{LaTeX y las ecuaciones}: uno de los motivos por los que la
  gente se pasa a LaTeX son las ecuaciones. Aquí hay una introducción
  sobre símbolos, comandos y demás parafernalia. Aprovecho para hablar
  un poco de la gestión del espacio y las referencias cruzadas.
\item
  \textbf{A vueltas con el idioma}: sobre paquetes de idioma y
  codificación de entrada y de fuente. Diferencias según el compilador
  que estemos usando.
\item
  \textbf{Formas, tamaños y colores}: hablamos del formato, en
  particular de cómo cambiar el tamaño, el estilo y el color del texto.
\item
  \textbf{La página}: sobre la posición de los elementos dentro de la
  página. Nos centramos en los márgenes, el interlineado y la
  alineación.
\item
  \textbf{Espacio en blanco}: sobre la gestión del espacio vertical y
  horizontal. Hablamos de párrafos, salto de línea y página y de cómo
  los trata LaTeX.
\item
  \textbf{Un documento científico}: vemos cuáles sonlas partes de un
  documento largo y su formato. Información sobre las unidades, el
  glosario y la bibliografía.
\item
  \textbf{Píntame ese código}: sobre resaltado de sintaxis. Hablamos de
  los paquetes \texttt{listings} y \texttt{minted}.
\item
  \textbf{También podemos presentar}: vemos que se pueden hacer
  presentaciones en LaTeX. Hablamos de las diferentes opciones y de
  programas compatibles.
\item
  \textbf{Nuestras propias macros}: creamos nuestros propios comandos y
  entornos y modificamos los que ya existen.
\item
  \textbf{Abramos la caja de herramientas}: hablamos de herramientas
  variadas que nos ayudan en nuestro proceso de escritura. Tratamos los
  borradores, las plantillas, el control de versiones y otras
  herramientas externas interesantes.
\item 
  \textbf{La opción Pandoc}:
\item
  \textbf{Mi proceso de escritura}:
\end{enumerate}

\subsection*{Apéndices}

\begin{enumerate}
\item
  \textbf{Una nota sobre los archivos auxiliares}: vemos para qué sirven
  todos esos archivos que LaTeX nos deja bailando por ahí.
\item
  \textbf{Hablemos de paquetes}: unas notas sobre la instalación de
  paquetes, especialmente centradas en GNU/Linux.
\end{enumerate}

\section*{Agradecimientos}\label{sec:agradecimientos}

Hay que agradecer el nacimiento de este curso a tres personas: los dos
profesores que me dieron clase de LaTeX, que han sido los peores
profesores que yo he tenido jamás (y he tenido muchos profesores malos,
creedme) y al genio entre los genios que hizo el mismo documento desde
cero en LaTeX y Word porque no sabía que se podía pasar de uno a otro (y
tampoco sabía buscar en Internet, entiendo) y se siente tan orgulloso de
sí mismo por ello que lo va pregonando por ahí.

Aparte, y ahora ya en serio, gracias a todos los que habéis leído cada
capítulo de este curso, habéis comentado y aportado y me habéis animado
a seguir.
