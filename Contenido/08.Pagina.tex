En el capítulo anterior estuvimos hablando de los diferentes estilos que
le podemos dar al texto, ahora nos centraremos en el formato de la
propia página. Vamos a ver cómo configurar cuatro cosas que los que
venimos de editores tipo Libre Office echamos de menos: la alineación,
el interlineado, los márgenes y la sangría. Va un
\href{https://www.xkcd.com/378/}{\emph{spoiler}}:

\begin{quote}
There is a LaTeX package for it
\end{quote}

\section{Alineación}\label{sec:alineacion}

Os habréis fijado que LaTeX justifica el texto por defecto, algo que a
mí personalmente me encanta. Para cambiar la alineación tenemos comandos
y entornos equivalentes, los comandos hacen efecto en el grupo actual y
los entornos en su contenido. Tanto los unos como los otros valen tanto
para el texto como para las tablas y figuras:

\begin{itemize}
\itemsep1pt\parskip0pt\parsep0pt
\item
  Para \textbf{alinear a la izquierda} tenemos el entorno
  \lstinline!flushleft! (\emph{nivelado a la izquierda}) y el comando
  \lstinline!\raggedright! (\emph{desigual a la derecha})
\item
  Para \textbf{alinear a la derecha} podemos elegir entre el entorno
  \lstinline!flushright! (\emph{nivelado a la derecha}) y el comando
  \lstinline!\raggedleft! (\emph{desigual a la izquierda})
\item
  Para \textbf{centrar} podemos usar el entorno \lstinline!center! o el
  comando \lstinline!\centering!
\end{itemize}

Hay dos diferencias entre usar un entorno y un comando para alinear:

\begin{itemize}
\item
  El entorno añade espacio blanco adicional delante y detrás de su zona
  de aplicación
\item
  El comando necesita un comando que le diga dónde acaba el párrafo, es
  por ello que se usa dentro de entornos como \lstinline!figure! o
  \lstinline!quote!. En un caso general tendremos que añadir una línea
  en blanco o el comando \lstinline!\par! al final del párrafo.
\end{itemize}

Los comandos son bastante prácticos cuando ya estamos dentro de un
entorno porque nos evitamos escribir otro \lstinline!begin! y
\lstinline!end! y, además, mejoramos la legibilidad.

Van unos ejemplos de texto alineado a la derecha:

\begin{lstlisting}[language={[latex]tex}]

% Texto alineado a la derecha

% Opción 1: entorno flushright
\begin{flushright}
  Texto
\end{flushright}

% Opción 2: \raggedleft + \par
{\raggedleft Texto\par}

% Opción 3: \raggedleft + línea en blanco
{\raggedleft Texto

}

% Opción 4: \raggedleft + entorno que indica final de párrafo
\begin{quote}
  \raggedleft
  Texto
\end{quote}
\end{lstlisting}

Acabo de descubrir que en español se puede uno referir a la alineación
como \href{http://glosariografico.com/bandera}{\emph{bandera}}, es
decir, un texto alineado a la derecha tendrá \emph{bandera a la
derecha}, porque parece una bandera cuyo mástil está a la derecha. Qué
cosas que aprendemos, oiga.

\section{Interlineado}\label{sec:interlineado}

Antes de deciros cómo cambiar el interlineado os dejo con este extracto
del artículo
\href{http://www.tex.ac.uk/FAQ-linespace.html}{\emph{Double-spaced
documents in LaTeX}} de \href{http://www.tex.ac.uk/index.html}{la lista
de preguntas frecuentes de LaTeX\footnote{También os podría contar como
  tuve que escribir mi Proyecto de Investigación en Arial 12, con
  interlineado de 1.5 y unos márgenes que daban ganas de llorar a pesar
  de que estaba usando LaTeX. Todo ello porque había unas
  \emph{exigencias de formato}, algo muy genial teniendo en cuenta que
  ese documento pasaba por un supuesto comité en el que nadie se lo
  leía. \href{http://writer2latex.sourceforge.net/}{Writer2Latex} me
  salvó la vida ahí. Pero mejor lo dejamos para otro día.}}:

\begin{quote}
Of course, the real solution (other than for private copy editing) is
not to use double-spacing at all. Universities, in particular, have no
excuse for specifying double-spacing in submitted dissertations: LaTeX
is a typesetting system, not a typewriter-substitute, and can (properly
used) make single-spaced text even more easily readable than
double-spaced typewritten text.
\end{quote}

Se refiere al interlineado doble, pero lo mismo me vale para cualquiera
que no sea el simple. Pero, como todo en LaTeX, lo podemos cambiar si
nos da por ahí, qué demonios. Para ello según ese mismo documento, lo
mejor es usar el paquete
\href{https://www.ctan.org/pkg/setspace}{\lstinline!setspace!} ya que
mantiene el interlineado simple en los pies de figura y tabla o en las
notas al pie, sitios donde no nos aporta nada que las líneas estén más
separadas.

Simplemente cargamos el paquete y elegimos el interlineado:

\begin{lstlisting}[language={[latex]tex}]
% Preámbulo
\usepackage{setspace}

\doublespacing % Interlineado doble
% \onehalfspacing % Interlineado 1.5
% \singlespacing % Interlineado simple
\end{lstlisting}

\section{Márgenes}\label{sec:margenes}

Para los márgenes, como en los demás casos, lo mejor es usar un paquete
que gestione las cosas por nosotros. Yo uso
\href{http://ctan.org/pkg/geometry}{\lstinline!geometry!} porque es el
que usa Pandoc y yo confío ciegamente en él.

Tan fácil de usar como cargar el paquete dándole como argumento opcional
el tamaño del margen. Para un margen uniforme haríamos:

\begin{lstlisting}[language={[latex]tex}]
\usepackage[margin=1cm]{geometry}
\end{lstlisting}

Y para definir cada margen por su lado:

\begin{lstlisting}
\usepackage[top=1cm, bottom=1cm, right=0.5cm, left=1.5cm]{geometry}
\end{lstlisting}

Para más detalles siempre está disponible
\href{http://osl.ugr.es/CTAN/macros/latex/contrib/geometry/geometry.pdf}{el
manual}

\section{Sangría}\label{sec:sangria}

Las normas tipográficas nos dicen que para separar dos párrafos podemos
usar una
\href{http://practicaltypography.com/first-line-indents.html}{línea en
blanco o sangría},
\href{https://www.youtube.com/watch?v=LbDMJ5YMaxM}{pero no ambas cosas}.
Si no le decimos nada, LaTeX se decanta por la segunda opción. Para
separar los párrafos con líneas en blanco lo más fácil es usar el
paquete \href{http://ctan.org/pkg/parskip}{\lstinline!parskip!}. También
podríamos poner \lstinline!\noindent! delante de cada párrafo que no
queremos que se indente o usar \lstinline!\setlength{\parindent}{0cm}!,
pero el paquete \lstinline!parskip! afecta al documento completo y nos
evita problemas con la gestión del espacio.

Lo cargamos en el preámbulo y a correr:

\begin{lstlisting}[language={[latex]tex}]
\usepackage{parskip}
\end{lstlisting}

\section{Recapitulemos}\label{sec:recapitulemos}

Como recomendación general a la hora de tocar el formato:

\begin{quote}
Lo mejor es dejarle siempre a LaTeX las decisiones de estilo. Él sabe de
tipografía y nosotros no, si hurgamos la probabilidad de convertir el
documento en una cosa que dé dolor de ojos es muy alta.
\end{quote}

Y como recapitulación a lo que hemos visto sobre la página:

\begin{itemize}
\item
  \textbf{Alineación}: LaTeX justifica por defecto pero podemos cambiar
  el alineado con entornos y comandos, \lstinline!flushleft! y
  \lstinline!\raggedright! alinean a la izquierda;
  \lstinline!flushright! y \lstinline!\raggedleft! a la derecha y
  \lstinline!center! y \lstinline!\centering! al centro.
\item
  \textbf{Interlineado}: de por sí es simple y el paquete
  \lstinline!setspace! nos ayuda a cambiarlo mediante órdenes como
  \lstinline!\doublespacing!.
\item
  \textbf{Márgenes}: se pueden modificar con el paquete
  \lstinline!geometry! mandándole los valores de los márgenes al paquete
  como argumento opcional.
\item
  \textbf{Sangría}: LaTeX sangra la primera línea de cada párrafo
  (excepto el primero) pero podemos modificar este comportamiento con el
  paquete \lstinline!parskip!.
\end{itemize}

\section{Referencias}\label{sec:referencias8}

\href{https://en.wikibooks.org/wiki/LaTeX/Page_Layout}{\emph{LaTeX/Page
Layout} en WikiBooks}

\href{http://texblog.org/2011/09/30/quick-note-on-line-spacing/}{\emph{Quick
note on line spacing}}

\href{https://en.wikibooks.org/wiki/LaTeX/Paragraph_Formatting\#Paragraph_indent_and_break}{\emph{Paragraph
indent and break} en WikiBooks}

\href{https://www.phy.duke.edu/~rgb/General/latex/ltx-300.html}{\lstinline!\\raggedleft!
en Hypertext Help with LaTeX}
