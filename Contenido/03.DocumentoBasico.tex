En este capítulo vamos a crear nuestro primer documento, muy
emocionante todo. Pero antes tenemos que hablar de estructura,
entornos y sintaxis. ¡Ánimo!

\section{Un documento y sus partes}\label{un-documento-y-sus-partes}

Un documento escrito en LaTeX tiene esta pinta\footnote{Este ejemplo
  está adaptado del
  \href{https://github.com/ekaitz-zarraga/programming-notes}{repo de
  apuntes de programación} de mi señor hermano y mío. Es una versión
  simplificada.}:

\begin{lstlisting}[language={[latex]tex}, caption={Ejemplo de documento escrito con \LaTeX}]

\documentclass[a4paper,10pt]{article}

% PREÁMBULO 

% Paquetes

\usepackage[utf8]{inputenc}
\usepackage[spanish,es-tabla]{babel}
\usepackage[T1]{fontenc}

\usepackage{listings}

% Comandos

\renewcommand{\lstlistingname}{Código}
\renewcommand{\lstlistlistingname}{Índice de fragmentos de código fuente}

% Opciones

\title{Python 2.*}
\author{Ondiz Zarraga}

%%%%%%%%%%%%%%%%%%%%%%%

\begin{document}
\maketitle

\begin{abstract}
Este documento es una pequeña guía de Python 
\end{abstract}

\tableofcontents

\section{Sobre el lenguaje}

\begin{itemize}
    \item Interpretado
    \item Indentación obligatoria
    \item Distingue mayúsculas - minúsculas
    \item No hay declaración de variables (\textit{dynamic typing})
    \item Orientado a objetos  
    \item Garbage colector: quita los objetos a los que no haga referencia nada
\end{itemize}

\end{document}
\end{lstlisting}

El documento tiene dos cosas llamativas:

\begin{itemize}
\item
  Cosas que empiezan por contrabarra, los \textbf{comandos}
\item
  Cachos que van entre un \lstinline!\begin! y un \lstinline!\end!, los
  \textbf{entornos}
\end{itemize}

Los comandos son la manera en la que nos comunicamos con LaTeX y un
entorno es el pedazo donde se aplica un formato.

Un entorno superimportante es el entorno \lstinline!document!, ahí
dentro irá el contenido de nuestro documento. Todo lo que va entre la
definición de documento (\lstinline!\documentclass!) y el inicio del
entorno \lstinline!document! se conoce como \textbf{preámbulo} y es
donde se cargan paquetes (\lstinline!\usepackage!), se definen comandos
y se establecen opciones.

\section{Nuestro primer documento}\label{nuestro-primer-documento}

Ahora que sabemos cómo es un documento vamos a empezar a escribir uno.
El documento más básico que podemos hacer es este:

\begin{lstlisting}[language={[latex]tex}]
\documentclass{article}

\begin{document}
Hola
\end{document}
\end{lstlisting}

Esto podemos compilarlo con el botoncillo de compilar de nuestro IDE, en
la terminar o con Pandoc.

Para compilar en la terminal haríamos (imaginemos que nuestro documento
se llama \lstinline!hola.tex!):

\begin{lstlisting}[language=bash]
pdflatex hola.tex
\end{lstlisting}

Evidentemente podéis compilar también con \lstinline!xelatex! de manera
equivalente. Si quisiéramos utilizar Pandoc tendríamos que hacer lo
siguiente:

\begin{lstlisting}[language=bash]
pandoc hola.tex -o hola.pdf
\end{lstlisting}

En los tres casos el resultado es el mismo excepto porque Pandoc no nos
deja
\href{https://ondahostil.wordpress.com/2016/11/17/lo-que-he-aprendido-archivos-auxiliares-de-latex/}{archivos
auxiliares} por ahí bailando.

¡Ya hemos escrito un documento! Debo reconocer que no es un documento
demasiado interesante, para hacer algo más chulo tenemos que aprender un
poco más de sintaxis.

\section{Un poco de sintaxis}\label{un-poco-de-sintaxis}

Para poder escribir un documento un poco más interesante tenemos que
aprendernos unos pocos comandos, hoy os voy a hablar de algunos
sencillos para que le vayáis cogiendo el truco, en las siguientes
entregas entraremos más en detalle.

Antes de nada una cosa, aun no hemos aprendido a establecer las opciones
de idioma y por lo tanto tendremos problemas en los idiomas con acentos.
Si queréis poneros a jugar ya, ponedme esto en el preámbulo, justo
debajo de \lstinline!\documentclass!:

\begin{lstlisting}[language={[latex]tex}]
\usepackage[spanish]{babel} % sustituir spanish por el idioma
\usepackage[utf8]{inputenc}
\usepackage[T1]{fontenc}
\usepackage{lmodern}
\end{lstlisting}

En el Capítulo \ref{ch:idioma} analizaremos mejor este tema.

\subsection{Título, capítulos y
secciones}\label{tuxedtulo-capuxedtulos-y-secciones}

Una cosa importante de LaTeX es que nos desacopla el contenido del
documento de su formato. Con esto quiero decir que nosotros le diremos
cuál es el título del documento y dónde comienzan las secciones y él les
dará el formato correspondiente según el tipo de documento y las
opciones que hayamos establecido previamente.

Pongamos por ejemplo los tipos de documento \lstinline!article! y
\lstinline!book!. Como su nombre indica, el primero se utiliza para
escribir artículos y el segundo libros. Como LaTeX es muy listo, cuando
le digamos que escriba el título, para el caso del artículo nos lo
escribirá en la parte superior de la página con el texto debajo, pero
para el caso del libro nos creará una portada. Para ambos casos la
sintaxis es exactamente la misma:

\begin{lstlisting}[language={[latex]tex}]
\documentclass{article} % O book

% Definimos el título
\title{Título del documento}

\begin{document}
\maketitle % Creamos el título
\end{document}
\end{lstlisting}

Del mismo modo, nosotros solo le tenemos que decir el título de la
sección o el capítulo y él le dará el formato correspondiente. Otra
diferencia entre las clases \lstinline!article! y \lstinline!book! es
que \lstinline!article! no tiene capítulos, como es lógico.

Para definir capítulos y secciones utilizamos los comandos
\lstinline!\chapter! y \lstinline!\section! en el cuerpo del documento,
es decir, después de \lstinline!\begin{document}!. Por ejemplo:

\begin{lstlisting}[language={[latex]tex}]
\chapter{Capítulo numerado}
\section{Primera sección}
\subsection{Primera subsección}

\chapter*{Capítulo sin numerar}
\section*{Sección sin numerar}
\end{lstlisting}

Como veis, podemos usar los comandos de sección y capítulo con el
asterisco para que no nos las numere. Otra cosa interesante es el grado
de anidación, tenemos secciones, subsecciones, subsubsecciones, párrafos
(\lstinline!\paragraph!) y subpárrafos (\textbackslash{}subparagraph),
cada uno con su formato definido. La clase \lstinline!book!\footnote{La
  clase \lstinline!book! no es la única que tiene partes y tal, pero de
  momento así nos vale.} también tiene por encima de las secciones,
capítulos y partes (\lstinline!\part!). Más adelante aprenderemos a
personalizar todos estos formatos porque si hay algo bueno que tiene
LaTeX es que nos deja cambiar \emph{absolutamente todo} y con relativa
facilidad (gracias a StackExchange, especialmente).

\subsection{Listas}\label{listas}

Si sois como yo os gustará especialmente este apartado: las listas.
LaTeX tiene dos tipos de listas: las numeradas y las sin numerar. Son
respectivamente los entornos \lstinline!enumerate! e
\lstinline!itemize!. Se usan exactamente igual, así que solo pongo el
ejemplo de uno de ellos:

\begin{lstlisting}[language={[latex]tex}]
\begin{itemize}
  \item Primer ítem
  \item Segundo ítem
\end{itemize}
\end{lstlisting}

Lo mejor es que podemos mezclar y anidar estos dos entornos que el
simbolito y la indentación cambiarán solos sin que nos tengamos que
preocupar. Por ejemplo,

\begin{lstlisting}[language={[latex]tex}]
\begin{enumerate}
    \item Primer ítem
    \item Segundo ítem con subítems
    \begin{itemize}
        \item Ítem sin numerar
    \end{itemize}
    \item Tercer ítem
\end{enumerate}
\end{lstlisting}

Quedará así:

\begin{center}\rule{0.5\linewidth}{\linethickness}\end{center}

\begin{enumerate}
\item
  Primer ítem
\item
  Segundo ítem con subítems

  \begin{itemize}
  \item
    Ítem sin numerar
  \end{itemize}
\item
  Tercer ítem
\end{enumerate}

\begin{center}\rule{0.5\linewidth}{\linethickness}\end{center}

\subsection{Imágenes}\label{imuxe1genes}

Hoy solo vamos a ver como colocar una única imagen, que lo de las
imágenes tiene un poco de lío. Lo que debemos saber es lo siguiente:

\begin{itemize}
\item
  Necesitamos llamar al paquete \lstinline!graphicx! en el preámbulo.
  Para eso escribiremos \lstinline!\usepackage{graphicx}! entre
  \lstinline!\documentclass! y \lstinline!\begin{document}!
\item
  Las imágenes se insertan con el comando
  \lstinline!\includegraphics[opciones]{ruta}!
\item
  Si queremos ponerles un pie de figura y una etiqueta, decidir su
  posición en la página y demás necesitamos el entorno
  \lstinline!figure!
\end{itemize}

Aquí tenemos un ejemplo de cómo insertar una imagen con en el entorno
\lstinline!figure!:

\begin{lstlisting}[language={[latex]tex}]
\begin{figure}[h] % opción de posicionamiento
    \caption{Pie de imagen}
    \centering % imagen centrada
    % Imagen 50% de ancho del texto
    \includegraphics[width=0.5\textwidth]{ruta_a_la_imagen}
\end{figure}
\end{lstlisting}

Una cosa importante de LaTeX son los objetos \emph{flotantes}, es decir,
los que si no les obligamos, se colocan en el hueco que mejor les venga
del documento. Esto es lo que nos ocurre con las imágenes al usar el
entorno \lstinline!figure!. Nos ayuda a que no haya huecos chungos en el
documento pero a veces junta todas las imágenes en una misma página o al
final del documento. Para evitar esto tenemos las opciones de
posicionamiento, de las que hablaremos cuando profundicemos en las
imágenes.

\subsection{Tablas}\label{tablas}

Para mí las tablas son lo peor de todo LaTeX. Son la cosa menos amigable
que se puede echar uno a la cara. Con un IDE la cosa mejora, pero
imaginaros cómo será el tema que la mitad de las veces las creo online
\href{http://www.tablesgenerator.com/}{aquí} y luego pego el resultado.

Lo que debemos de saber de las tablas es lo siguiente:

\begin{itemize}
\item
  El contenido se especifica con el entorno \lstinline!tabular!.
  Separemos las celdas con el ampersand y cambiaremos de línea con dos
  contrabarras.
\item
  Si queremos ponerles un pie de tabla y una etiqueta, decidir su
  posición en la página y demás necesitamos el entorno \lstinline!table!
\item
  Si utilizamos el entorno \lstinline!table! la tabla se volverá
  \emph{flotante}
\end{itemize}

Aquí tenemos un ejemplo de tabla sencilla:

\begin{lstlisting}[language={[latex]tex}]
\begin{table}
    \begin{tabular}{|ll|}
      \hline % Separador
      Columna 1 & Columna 2 \\
      1         & 2         \\
      3         & 4         \\
      \hline
    \end{tabular}
    \caption {Pie de tabla}
\end{table}
\end{lstlisting}

Al igual que con la imágenes, profundizaremos en las tablas más
adelante.

\subsection{Ecuaciones}\label{ecuaciones}

Las ecuaciones son la razón por la que mucha gente se pasa a LaTeX. Son
muy cucas y escribirlas, una vez cogido el callo, no es un infierno (de
nuevo, ¡hola, Word!). Pero no nos vamos a engañar, al principio es
\emph{muerte y destrucción}. Hay dos tipos de ecuaciones en LaTeX: las
que van dentro de la línea y las que tienen línea propia. Las primeras
van entre signos de dólar y las segundas dentro del entorno
\lstinline!equation!. Aquí tenemos un ejemplo:

\begin{lstlisting}[language={[latex]tex}]
Imaginemos que $a=1$ en la siguiente ecuación:
\begin{equation}
ax^2 + 1 = 0
\end{equation}
\end{lstlisting}

Del mismo modo que ocurría con las secciones, si utilizamos el asterisco
la ecuación no estará numerada.

El lío con las ecuaciones es que todo se define con comandos, por
ejemplo, \lstinline!\frac{numerador}{denominador}! se usa para escribir
una fracción y \lstinline!\omega! para la letra griega omega. Los que
uséis un IDE lo tenéis más fácil porque suelen tener una barrita con los
símbolos más usados, a los demás les tocará investigar.

También tenemos otras herramientas que nos pueden ayudar a escribir
ecuaciones:

\begin{itemize}
\item
  \textbf{Editores online}: son editores con su GUI y tal que nos
  facilitan la tarea cuando todavía no conocemos los comandos de LaTeX,
  por ejemplo tenemos
  \href{http://www.numberempire.com/texequationeditor/equationeditor.php\%22}{Latex
  Equation Editor} o
  \href{https://www.latex4technics.com/}{LaTeX4technics}.
\item
  \href{http://detexify.kirelabs.org/classify.html}{\textbf{Detexify}}:
  un cacharro que nos busca el símbolo de LaTeX más parecido a algo que
  le pintemos.
\end{itemize}

Las ecuaciones se merecen una entrada propia y la tendrán.

\section{Bonus: opción Markdown}\label{bonus-opciuxf3n-markdown}

Si tenemos instalado Pandoc todo lo que hemos explicado aquí podemos
conseguirlo escribiendo en Markdown
(\href{http://rmarkdown.rstudio.com/authoring_pandoc_markdown.html}{sabor
Pandoc}) y convirtiendo a pdf. Por ejemplo, podemos meter una imagen
como:

\begin{lstlisting}
![Pie de imagen](/ruta){width=0.7 #etiqueta}
\end{lstlisting}

\section{Referencias}\label{referencias}

\href{https://en.wikibooks.org/wiki/LaTeX/Document_Structure}{\emph{LaTeX/Document
Structure} en Wikibooks}

\href{https://www.sharelatex.com/learn/Environments}{\emph{Environments}
en ShareLaTeX}

\href{http://www.personal.ceu.hu/tex/environ.htm}{\emph{Latex Standard
Environments}}

\href{http://texblog.org/2013/02/13/latex-documentclass-options-illustrated/}{\emph{LaTeX
documentclass options illustrated}}

\href{https://www.sharelatex.com/learn/Sections_and_chapters}{\emph{Sections
and chapters} en ShareLaTeX}

\href{https://en.wikibooks.org/wiki/LaTeX/Tables}{\emph{LaTeX/Tables} en
Wikibooks}

\href{https://www.sharelatex.com/learn/Inserting_Images}{\emph{Inserting
images} en ShareLaTeX}

\href{http://osl.ugr.es/CTAN/info/symbols/comprehensive/symbols-a4.pdf}{\emph{The
Comprehensive LaTeX Symbol List}}

\href{ftp://ftp.ams.org/pub/tex/doc/amsmath/short-math-guide.pdf}{\emph{Short
Math Guide for LaTeX}}
