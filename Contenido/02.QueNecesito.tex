Voy a intentar explicar qué necesitamos para escribir antes de explicar
la sintaxis. Si no os va la chapa, pasad directamente a la
recapitulación.

\section{El editor y el compilador}\label{el-editor-y-el-compilador}

Antes de ponernos a hacer nada vamos a diferenciar dos cosas: el
\emph{editor} y el \emph{compilador}. Algo que para los que andáis en la
informática es megaevidente para el resto de nosotros oh mortales puede
suponer un lío bastante gordo.

Yo lo resumo así: puedes escribir tus historias de LaTeX en el Bloc de
Notas si quieres (\emph{el editor}), luego te buscas la vida para
convertirlo a algo que un ser humano pueda leer (\emph{el compilador}).

Ahora vamos a liarnos la manta. Resulta que LaTeX son unas macros
escritas para TeX\footnote{De hecho hay otras llamadas
  \href{https://en.wikipedia.org/wiki/ConTeXt}{ConTeXt} pero nos vamos a
  olvidar de ellas.} por lo que tenemos dos lenguajes de marcado que
además se pueden compilar con diferentes \textbf{compiladores}\footnote{Veréis
  a lo que yo llamo ``compilador'' por ahí también como \emph{LaTeX
  engine}}. Aquí tenéis un resumen rápido:

\begin{itemize}
\item
  \texttt{tex} y \texttt{latex}: compilan respectivamente TeX y LaTeX a
  dvi. Para los siguientes el que solo contenga \texttt{tex} compilará
  TeX y el que contenga \texttt{latex} compilará LaTeX
\item
  \texttt{pdftex}/\texttt{pdflatex}: compilan a pdf
\item
  \texttt{xetex}/\texttt{xelatex}: compilan a pdf pero tienen la
  diferencia que gestionan Unicode y pueden usar las fuentes del sistema
  sin necesidad de configurar nada.
\item
  \texttt{luatex}/\texttt{lualatex}: compilan a pdf. La diferencia es
  que están escritos en \href{http://www.lua.org/}{Lua}, un lenguaje de
  programación bastante interesante
\end{itemize}

Bien, ahora que sabemos de compiladores vamos a ver cómo conseguimos
nosotros tener uno que nos genere los documentos. Aquí entran las
\textbf{distribuciones} de LaTeX. Una distribución es un conjunto de
programas y paquetes que nos permiten escribir sin tener que configurar
todo a mano. Es decir, si instalamos una distribución tendremos los
compiladores de los que hablábamos antes, un gestor de paquetes y otras
cosas útiles. De los paquetes hablaremos más adelante, pero de momento
os puedo decir que las diferentes funcionalidades van en diferentes
paquetes para que podamos cargar solo las que nos interesen.

Las distribuciones más conocidas son estas:

\begin{itemize}
\item
  \href{http://www.tug.org/texlive/}{TeXLive}, distribución
  multiplataforma, la encontramos para GNU/Linux, Windows y MacOS.
\item
  \href{https://miktex.org/}{MikTeX}, una distribución específica para
  Windows
\end{itemize}

No voy a hablar de la instalación porque está más que documentada y es
sencillita (especialmente para mis hermanos linuxeros, que la tienen en
los repositorios).

Como no había suficiente locura, nos quedan los \textbf{editores}. En
sí, podemos escribir en cualquier programa pero yo personalmente no os
lo recomiendo. Al menos elegid uno que tenga sintaxis resaltada para que
no os quedéis birojos intentando descifrar qué es formato y qué
contenido. Podemos dividir los editores en dos grupos:

\begin{itemize}
\item
  \emph{Editores de propósito general}: son los que sirven para escribir
  en general. Van desde uno simple como
  \href{https://es.wikipedia.org/wiki/Gedit}{gedit} hasta bestias pardas
  como \href{http://www.vim.org/}{Vim} o mi muy amado
  \href{https://www.gnu.org/software/emacs/}{Emacs}. A nada de potente
  que sea el editor seguramente tendrá un modo o un plugin que nos
  permita compilar también.
\item
  \emph{Editores específicos}
  (\href{https://es.wikipedia.org/wiki/Entorno_de_desarrollo_integrado}{IDE}):
  son los editores desarrollados expresamente para escribir LaTeX. Hay
  bastantes, yo he usado
  \href{http://texstudio.sourceforge.net/}{TeXstudio} en Windows y
  \href{http://kile.sourceforge.net/}{Kile} en GNU/Linux, pero no por
  una razón particular.
\end{itemize}

\section{¿Qué me conviene?}\label{quuxe9-me-conviene}

Después de el rollo que os he soltado diréis ¿y ahora qué demonios uso?
¿Me conviene un IDE o no? Pues a eso no os puedo responder directamente
porque depende de vuestra manera de trabajar y vuestra experiencia, esto
es lo que yo me plantearía:

\begin{itemize}
\item
  Si ya estáis usando un editor tipo Vim o Emacs, yo miraría su modo o
  plugin correspondiente antes de nada. Así tendremos las ventajas de
  usar un editor específico y las de usar un editor general.
\item
  Si os gusta tener todo centralizado y darle solo a un botoncillo para
  que se genere el documento final, un IDE es lo vuestro.
\item
  Si os gusta tener todo bajo control, no tenéis miedo de escribir un
  Makefile y no os apetece instalar otro programa en el ordenador (¡y
  menos uno con GUI!), podéis escribir en cualquier sitio y escribir las
  órdenes de compilar a mano. Eso sí, preparaos para leer manuales a
  mansalva.
\end{itemize}

Yo tengo que reconocer que soy más de las dos primeras opciones, pero me
parece justo decir que la tercera también existe y seguro que hay gente
que la prefiere.

\section{La opción Pandoc}\label{la-opciuxf3n-pandoc}

\href{http://pandoc.org/}{Pandoc} es, aparte del programa con el mejor
nombre de la historia, un \emph{conversor de documentos}, es decir,
puede convertir documentos de un formato a otro alegremente. Podemos
usarlo para no tener que usar un IDE y que se ocupe él de compilarnos el
documento. Sobre Pandoc hablaremos en el futuro, de momento simplemente
me vale con que sepáis que existe y no vayáis por ahí diciendo que
\emph{tengo escribir en Word porque me obliga mi jefe}, tendrás que
entregarle un \emph{doc}, pero escribirlo lo escribes donde te dé la
gana, faltaría más.

Como cosa curiosa, resulta que Pandoc usa LaTeX como etapa intermedia
para pasar de Markdown a pdf con lo que podemos aprovecharnos de la
sintaxis simple de Markdown y de la potencia de LaTeX simultáneamente.
Así es como he escrito yo mi tesis, de hecho. La desventaja, claro, es
que tenemos que saber tanto LaTeX como Markdown.

\section{Recapitulación:}\label{recapitulaciuxf3n}

Resumiendo, para poder escribir cosillas en LaTeX necesitamos:

\begin{itemize}
\item
  Un \textbf{editor}, puede ser uno de uso general (como Emacs) o uno
  específico para LaTeX (como Kile). Si nuestro editor no es capaz de
  compilar directamente necesitaremos también una terminal.
\item
  Una \textbf{distribución} de LaTeX, será diferente según nuestro
  sistema operativo. La distribución incluirá diferentes compiladores.
\end{itemize}

Para la \emph{opción Pandoc} necesitamos:

\begin{itemize}
\item
  Pandoc (obviamente)
\item
  Un editor cualquiera y una terminal
\item
  Una distribución de LaTeX
\end{itemize}

\section{Referencias}\label{referencias}

\href{http://tex.stackexchange.com/questions/49/what-is-the-difference-between-tex-and-latex}{\emph{What
is the difference between TeX and LaTeX?} en StackExchange}

\href{https://en.wikibooks.org/wiki/LaTeX/Basics\#Compilation}{\emph{LaTeX/compilation}
en Wikibooks}

\href{https://en.wikipedia.org/wiki/XeTeX}{\emph{XeTeX} en la wiki}

\href{http://www.luatex.org/}{\emph{LuaTeX}}

\href{http://tex.stackexchange.com/questions/126206/why-choose-lualatex-over-xelatex\#126216}{\emph{Why
choose LuaLaTeX over XeLaTeX?} en StackExchange}

\href{http://tex.stackexchange.com/questions/13593/the-differences-between-tex-engines\#13601}{\emph{The
differences between TeX engines} en StackExchange}

\href{https://www.sharelatex.com/learn/Choosing_a_LaTeX_Compiler}{\emph{Choosing
a LaTeX Compiler}}

\href{http://www.tug.org/interest.html\#free}{\emph{Free TeX
implementations}}

\href{https://en.wikibooks.org/wiki/LaTeX/Installation}{\emph{LaTex/Installation}
en Wikibooks}

\href{https://en.wikipedia.org/wiki/Comparison_of_TeX_editors}{\emph{Comparison
of TeX editors} en la wiki}

\href{http://www.tex.ac.uk/FAQ-make.html}{\emph{Makefiles for LaTeX
documents} en \emph{UK List of TeX Frequently Asked Questions}}
