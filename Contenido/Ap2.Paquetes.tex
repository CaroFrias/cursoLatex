En todo el curso no he hablado de instalación pero me parecía necesaria
una pequeña nota sobre los paquetes, especialmente para el caso de
GNU/Linux, ya que la distribución de LaTeX para Windows, MikTeX, instala
los paquetes necesarios de manera automática.

El caso de TeXLive es diferente porque hay dos maneras\footnote{También
  se pueden descargar los paquetes de CTAN y descomprimirlos en la
  carpeta correcta de nuestro LaTeX tal y como cuentan
  \href{https://en.wikibooks.org/wiki/LaTeX/Installing_Extra_Packages}{aquí},
  nunca lo he hecho y me parece un poco lioso.} de instalarlo en
GNU/Linux:

\begin{itemize}
\item
  \textbf{Desde los repositorios de nuestra distro}, cuando queramos
  instalar paquetes usaremos también los repositorios.
\item
  \textbf{Descargándolo
  \href{https://www.tug.org/texlive/doc/texlive-en/texlive-en.html\#installation}{por
  ahí}}, lo que suelen llamar \emph{TeXLive nativo}. En este caso habrá
  que usar el gestor de paquetes de TeXLive.
\end{itemize}

\section{Descargar paquetes desde los repositorios}

Los paquetes de TeXLive viven en paquetes de los repositorios (de
Debian, en mi caso\footnote{Creo que funciona de la misma manera en
  todas las distros, tenéis información específica
  \href{http://tug.org/texlive/distro.html}{aquí}.}) pero \emph{no
directamente con su nombre}. Me explico con un ejemplo: el soporte de
idioma.

Sabemos que para que LaTeX nos haga bien la silabación y que use las
palabras claves (\emph{Capítulo}, \emph{Sección}, \ldots{}) en el idioma
correspondiente necesitamos el paquete \lstinline!babel! con el idioma
como opción:

\begin{lstlisting}[language={[latex]tex}]
\usepackage[spanish]{babel}
\end{lstlisting}

Si hemos instalado la versión sencilla de TeXLive y añadimos esa línea a
nuestro \emph{tex} nos dará error al compilar porque nos falta el
paquete de español. Para buscarlo hacemos\footnote{Para ver cuál es el
  comando para buscar paquetes equivalente a
  \lstinline!apt-cache search! en otras distros podéis usar
  \href{https://colaboratorio.net/gestor-paquetes.html}{esta tabla} de
  los \emph{compis} de \href{https://colaboratorio.net/}{Colaboratorio}.}:

\begin{lstlisting}[language=bash]
apt-cache search texlive spanish
\end{lstlisting}

Veremos lo siguiente:

\begin{lstlisting}[language=bash]
texlive-latex-extra - TeX Live: LaTeX additional packages
texlive-doc-es - TeX Live: transitional dummy package
texlive-lang-spanish - TeX Live: Spanish
\end{lstlisting}

Esto nos da una pista de donde vive el paquete: en el grupo de paquetes
adicionales (\lstinline!texlive-latex-extra!) o en uno específico de
idioma (\lstinline!texlive-lang-spanish!). Ahora podemos instalar el que
prefiramos.

Resumiendo: \emph{cada vez que necesitemos un paquete de TeXLive debemos
buscar el paquete de los repositorios que lo contiene}. Este sistema es
cómodo pero tiene el problema de que los paquetes suelen ser ancianos,
de ahí la utilidad del segundo método.

\section{El gestor de paquetes de
TeXLive}

También se pueden instalar los paquetes con \lstinline!tlmgr!, el gestor
de paquetes de TeXLive. Para usar esta opción hay que instalar TeXLive
\href{http://tug.org/texlive/acquire-netinstall.html}{descargándolo de
la red}\footnote{Kile por ejemplo depende de TeXLive así que solo con
  una \emph{instalación nativa} no funcionará. En algunas distros está
  disponible el paquete \lstinline!texlive-dummy! para solucionar este
  problema. En Debian y derivados hay que instalar un
  \href{http://tug.org/texlive/debian.html\#vanilla}{\emph{vanilla
  TeXLive}}.}. Luego instalamos los paquetes escribiendo en la
terminal\footnote{¡También tiene una
  \href{https://darrengoossens.wordpress.com/tag/gui/}{GUI}!}:

\begin{lstlisting}[language=bash]
tlmgr install PAQUETE
\end{lstlisting}

Incluso podemos actualizar todos los paquetes instalados con:

\begin{lstlisting}[language=bash]
tlmgr update --all
\end{lstlisting}

Como siempre, tenemos toda la información sobre este comando en su
\href{https://www.tug.org/texlive/doc/tlmgr.html}{manual} y en este caso
también haciendo:

\begin{lstlisting}[language=bash]
tlmgr --help
\end{lstlisting}

Este sistema tiene la ventaja de que los paquetes que descarguemos
estarán actualizados, aparte de que tendremos acceso a paquetes que
todavía no son parte de los repositorios de nuestra distribución de
GNU/Linux.

\section{Referencias}

\href{http://tex.stackexchange.com/questions/28528/best-way-to-install-packages-for-texlive-in-ubuntu}{\emph{Best
way to install packages for TeXLive in Ubuntu?} en TeXExchange}

\href{https://en.wikibooks.org/wiki/LaTeX/Installing_Extra_Packages}{\emph{LaTeX/Installing
Extra Packages} en Wikibooks}

\href{https://www.tug.org/texlive/quickinstall.html}{\emph{TeX Live -
Quick install}}

\href{http://tex.stackexchange.com/questions/73526/how-to-install-a-language-package-in-texmaker-on-ubuntu-12-04\#73528}{\emph{How
to install a language package in Texmaker on Ubuntu 12.04?} en
TeXExchange}

\href{https://www.tug.org/texlive/pkginstall.html}{\emph{TeX Live
package installation}}

\href{http://tug.org/texlive/distro.html}{\emph{TeX Live and distros}}

\href{http://texblog.org/2011/05/12/updating-latex-tex-live/}{\emph{LaTeX
Installation} en texblog}

\href{https://tex.stackexchange.com/questions/114623/installing-texlive-on-ubuntu-revisited}{\emph{Installing
TeXlive on Ubuntu, revisited} en TeXExchange}

\href{https://wiki.archlinux.org/index.php/TeX_Live}{\emph{TeX Live} en
la wiki de Arch}

\href{https://www.tug.org/texlive/doc/tlmgr.html}{Manual de
\lstinline!tlmgr!}
